\PassOptionsToPackage{dvipsnames}{xcolor}
\PassOptionsToPackage{lowtilde}{url}

%% If you are using \orcid or academicons
%% icons, make sure you have the academicons
%% option here, and compile with XeLaTeX or LuaLaTeX.
% \documentclass[10pt,a4paper,academicons]{altacv}

%% Use the "normalphoto" option if you want a normal photo instead of cropped to a circle
% \documentclass[10pt,a4paper,normalphoto]{altacv}

\documentclass[10pt,a4paper,ragged2e, normalphoto]{altacv}

% Change the page layout if you need to
\geometry{left=1cm,right=9cm,marginparwidth=6.8cm,marginparsep=1.2cm,top=1.25cm,bottom=1.25cm}

% Change the font if you want to, depending on whether you're using pdflatex or xelatex/lualatex
\ifxetexorluatex
    \setmainfont{Carlito}
\else
    \usepackage[utf8]{inputenc}
    \usepackage[T1]{fontenc}
    \usepackage[default]{lato}
\fi

% Change the colours if you want to
\definecolor{IFBlue}{HTML}{0F6688}
\definecolor{IFDarkBlue}{HTML}{00425C}
\definecolor{IFMediumBlue}{HTML}{17739A}
\definecolor{IFLightBlue}{HTML}{228FBD}

\definecolor{IFGrey}{HTML}{6D6E71}
\definecolor{IFDarkGrey}{HTML}{231F20}
\definecolor{IFMediumGrey}{HTML}{4D4D4F}
\definecolor{IFLightGrey}{HTML}{808285}

\colorlet{heading}{IFBlue}
\colorlet{accent}{IFLightBlue}
\colorlet{emphasis}{IFMediumGrey}
\colorlet{body}{IFGrey}

\renewcommand{\itemmarker}{{\small\textbullet}}
\renewcommand{\ratingmarker}{\faCircle}

%% sample.bib contains your publications
\addbibresource{sample.bib}

\begin{document}
\name{Gustavo Pasqua de Oliveira Celani}
\tagline{Engenheiro de Computação}
%\photo{2.8cm}{logo}
\personalinfo{
    % \printinfo{symbol}{detail}
    \printinfo{\faCalendar}{06/03/1996}
    \email{gustavo\_celani@hotmail.com}
    \phone{+55 (35) 9 9152-0269}
    \location{Campinas - SP, Brasil}
}

%% Make the header extend all the way to the right, if you want.
\begin{fullwidth}
    \makecvheader
\end{fullwidth}

%% Depending on your tastes, you may want to make fonts of itemize environments slightly smaller
% \AtBeginEnvironment{itemize}{\small}

%%%%%%%%%%%%%%%%%%%%%%%%%%%%%%%%%%%%%%%%%%%%%%%%%%%%%%%%%%%%%%%%%%%%%%%%%%%%%%%%%%%%%%%%%%%%%%%%%%%%%
% Formação Acadêmica
%%%%%%%%%%%%%%%%%%%%%%%%%%%%%%%%%%%%%%%%%%%%%%%%%%%%%%%%%%%%%%%%%%%%%%%%%%%%%%%%%%%%%%%%%%%%%%%%%%%%%
\cvsection[page1sidebar]{Formação Acadêmica}

    \cvevent{Especialização - Segurança em Redes de Computadores}{Universidade Estadual de Campinas (UNICAMP)}{2019}{Campinas - SP, Brasil}{}{}

    \divider

    \cvevent{Bacharelado - Engenharia de Computação}{Instituto Nacacional de Telecomunicações (INATEL)}{2014 -- 2018}{Santa Rita do Sapucaí - MG, Brasil}{}{}
    \begin{itemize}
        \item Excelência Acadêmica
        \item Iniciação Científica - Centro de Desenvolvimento e Transferência de Tecnologia Assistiva (CDTTA)
        \item Monitoria - Cálculo I, Cálculo II, Cálculo III, Cálculo Numérico,\\Álgebra e Geometria Analítica, Probabilidade e Estatística
        \item Organizador do Workshop sobre Ferramentas Matemáticas\\Computacionais (MatLab e GeoGebra)
        \item Avaliador  de projetos no Seminário de Físical do INATEL (SEFITEL)
    \end{itemize}

%%%%%%%%%%%%%%%%%%%%%%%%%%%%%%%%%%%%%%%%%%%%%%%%%%%%%%%%%%%%%%%%%%%%%%%%%%%%%%%%%%%%%%%%%%%%%%%%%%%%%
% Experiência Profissional
%%%%%%%%%%%%%%%%%%%%%%%%%%%%%%%%%%%%%%%%%%%%%%%%%%%%%%%%%%%%%%%%%%%%%%%%%%%%%%%%%%%%%%%%%%%%%%%%%%%%%
\cvsection{Experiência Profissional}

    \cvevent{Engenheiro de Software Seguro}{Samsung Instituto de Desenvolvimento para Informática (SiDi)}{2018 - Atual}{Campinas - SP, Brasil}{}{}
    \begin{itemize}
        \item Samsung Security Platform (Knox System)
        \item Security Development Lifecycle (SDL)
        \item Security by Design
        \item Android (Aplicação, Framework, Kernel, TrustZone)
        \item Cloud Server Back-End
    \end{itemize}

    \divider

    \cvevent{Estágio - Engenheiro de Software Embarcado}{Leucotron Telecom}{2017 -- 2018}{Santa Rita do Sapucaí - MG, Brasil}{}{}
    \begin{itemize}
        \item Linux Embarcado
        \item Telecomunicações
        \item Session Initiation Protocol (SIP)
    \end{itemize}

    \divider

    \cvevent{Voluntário - Acessor do Núcleo de Gestão do Conhecimento}{CP2eJr Empresa Júnior do INATEL}{2014 -- 2016}{Santa Rita do Sapucaí - MG, Brasil}{}{}

    \divider

    \cvevent{Menor Aprendiz - Auxiliar Administrativo}{Infoserv Comércio \& Distribuição}{2012 -- 2014}{Guaxupé - MG, Brasil}{}{}

%%%%%%%%%%%%%%%%%%%%%%%%%%%%%%%%%%%%%%%%%%%%%%%%%%%%%%%%%%%%%%%%%%%%%%%%%%%%%%%%%%%%%%%%%%%%%%%%%%%%%
% Page Break
%%%%%%%%%%%%%%%%%%%%%%%%%%%%%%%%%%%%%%%%%%%%%%%%%%%%%%%%%%%%%%%%%%%%%%%%%%%%%%%%%%%%%%%%%%%%%%%%%%%%%
\newpage
\begin{fullwidth}

%%%%%%%%%%%%%%%%%%%%%%%%%%%%%%%%%%%%%%%%%%%%%%%%%%%%%%%%%%%%%%%%%%%%%%%%%%%%%%%%%%%%%%%%%%%%%%%%%%%%%
% Projetos
%%%%%%%%%%%%%%%%%%%%%%%%%%%%%%%%%%%%%%%%%%%%%%%%%%%%%%%%%%%%%%%%%%%%%%%%%%%%%%%%%%%%%%%%%%%%%%%%%%%%%
\cvsection{Projetos Pessoais}

    \cvevent{Criptografia de Curvas Elípticas: Definição e Recomendações}{Universidade Estadual de Campinas (UNICAMP)}{2019}{Campinas - SP, Brasil}{}{}

    \cvevent{Redes Neurais Artificiais Aplicadas à Geração e Troca de Chaves Criptográficas Binárias}{Instituto Nacional de Telecomunicações (INATEL)}{2018}{Santa Rita do Sapucaí - MG, Brasil}{}{}

    \cvevent{SensePro - Prótese Sensorial de Membro Superior}{Centro de Desenvolvimento e Transferência de Tecnologia Assistiva (CDTTA)}{2016}{Santa Rita do Sapucaí - MG, Brasil}{}{}

    \cvevent{OSMS - Órtese Sensorial de Membro Superior}{Centro de Desenvolvimento e Transferência de Tecnologia Assistiva (CDTTA)}{2015}{Santa Rita do Sapucaí - MG, Brasil}{}{}

%%%%%%%%%%%%%%%%%%%%%%%%%%%%%%%%%%%%%%%%%%%%%%%%%%%%%%%%%%%%%%%%%%%%%%%%%%%%%%%%%%%%%%%%%%%%%%%%%%%%%
% Publicações
%%%%%%%%%%%%%%%%%%%%%%%%%%%%%%%%%%%%%%%%%%%%%%%%%%%%%%%%%%%%%%%%%%%%%%%%%%%%%%%%%%%%%%%%%%%%%%%%%%%%%
\cvsection{Publicações}

    \cvevent{Adaptação de um Projeto de Robô Humanoide Impresso em 3D em uma Prótese Sensorial de Membro Superior}{2019 - Atena Editora: A Produção do Conhecimento na Engenharia Biomédica\\2016 - Congresso Brasileiro de Engenharia Biomédica (CBEB)}{}{}{https://www.atenaeditora.com.br/wp-content/uploads/2019/06/Ebook-A-Producao-do-Conhecimento-na-Engenharia-Biomedica.pdf}{http://www.sbeb.org.br/cbeb2016/pt/anais-e-certificados}

    \divider

    \cvevent{OSMS - Órtese Sensitiva de Membro Superior}{2016 - Congresso de Iniciação Científica do INATEL (INCITEL)\\2015 - Feira Tecnológica do INATEL (FETIN)}{}{}{http://www.inatel.br/incitel/anais-incitel/incitel-2016-1}{http://inatel.br/fetin/revista-atual/revista-vol-3-n-2}

%%%%%%%%%%%%%%%%%%%%%%%%%%%%%%%%%%%%%%%%%%%%%%%%%%%%%%%%%%%%%%%%%%%%%%%%%%%%%%%%%%%%%%%%%%%%%%%%%%%%%
% Cursos
%%%%%%%%%%%%%%%%%%%%%%%%%%%%%%%%%%%%%%%%%%%%%%%%%%%%%%%%%%%%%%%%%%%%%%%%%%%%%%%%%%%%%%%%%%%%%%%%%%%%%
\cvsection{Cursos}

\begin{itemize}
    \item 2020 - Cryptography I (Stanford University over Coursera - 30h)
    \item 2018 - Técnicas de Computação Forense (4Linux - 40h)
    \item 2018 - Fundamentos de Ethical Hacking (Udemy - 20h)
    \item 2018 - Programação Shell Script (Udemy - 18h)
    % \item 2018 - Hacking for Beginners (Udemy - 2h)
    \item 2017 - Lógica de Programação (Brasil Mais TI - 20h)
    \item 2017 - SQL Completo (Brasil Mais TI - 20h)
    \item 2017 - Leucotron Linha ISION IP Recursos Avançados (Leucotron - 40h)
    \item 2017 - Java Algoritmos (Softex - 40h)
    \item 2017 - Conceitos Básicos de Desenvolvimento de Software (Softex - 8h)
    \item 2017 - Introdução ao JavaScript (INATEL - 4h)
    \item 2017 - Conceitos de Desenvolvimento de Aplicações Híbridas com Ionic 2 (INATEL - 4h)
    \item 2016 - Lógica de Programação Orientação a Objetos utilizando Linguagem Swift (Hackatruck - 50h)
    \item 2016 - Ferramentas Gráficas: MatLab (INATEL - 18h)
    \item 2015 - Banco de Dados: MySQL (INATEL - 10h)
    \item 2015 - Capacitação Didática Comportamental para Monitores e PEDs (INATEL - 36h)
    \item 2015 - Programação Orientada a Objetos utilizando Java (INATEL - 16h)
    % \item 2015 - Desenvolvimento de Jogos em Unity 3D (INATEL - 4h)
    % \item 2014 - Desenvolvimento de Jogos em Construct 2 (INATEL - 20h)
    \item 2014 - Desenvolvimento Básico para Windows Phone (INATEL – 4h)
\end{itemize}

%%%%%%%%%%%%%%%%%%%%%%%%%%%%%%%%%%%%%%%%%%%%%%%%%%%%%%%%%%%%%%%%%%%%%%%%%%%%%%%%%%%%%%%%%%%%%%%%%%%%%
% Page End
%%%%%%%%%%%%%%%%%%%%%%%%%%%%%%%%%%%%%%%%%%%%%%%%%%%%%%%%%%%%%%%%%%%%%%%%%%%%%%%%%%%%%%%%%%%%%%%%%%%%%
\end{fullwidth}
\end{document}
